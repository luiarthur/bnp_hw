% To compile: pdflatex file.tex
\documentclass[11pt]{article}
\usepackage{fullpage}
\usepackage{pgffor}
\usepackage{amssymb}
\usepackage{bm}
\usepackage{mathtools}
\usepackage{verbatim}
\usepackage{appendix}
\usepackage{graphicx}
\usepackage{color}
\usepackage{subfig}
\usepackage{url} % for underscore in footnote
\usepackage[UKenglish]{isodate} % for: \today
\cleanlookdateon                % for: \today
\usepackage{natbib} % Can remove if no bibliography. bibtex
%\pagestyle{empty} % Removes page number. Graphs too big.

\def\wl{\par \vspace{\baselineskip}\noindent}
\def\beginmyfig{\begin{figure}[h]\center}
\def\endmyfig{\end{figure}}
\def\ds{\displaystyle}
\def\tu{\textunderscore}
\definecolor{grey}{rgb}{.2,.2,.2}
\definecolor{lgrey}{rgb}{.8,.8,.8}
\def\hline{ \textcolor{lgrey}{\hrulefill} }
\newcommand{\m}[1]{\mathbf{\bm{#1}}} % Serif bold math
\def\ds{\displaystyle}                                                    
\def\inv{^{\raisebox{.2ex}{$\scriptscriptstyle-1$}}}
\def\pm{^{\raisebox{.2ex}{$\scriptscriptstyle\prime$}}}
\def\norm#1{\left\lVert#1\right\rVert}

% \def for THIS ASSIGNMENT!!!%%%%%%%%%%%%%%%%%%%

\begin{document}
% my title:
\begin{center}
  {\huge \textbf{Review of the Spatial Dirichlet Process}
    \footnote{\url{https://github.com/luiarthur/bnp_hw/project/bnp_spatialDP}}
  }\\
  \wl
  UCSC AMS 241 Course Project\\
  \noindent\today\\
  Arthur Lui\\
  \hline
\end{center}

\noindent
This project is a review of the spatial Dirichlet process (SDP) developed by
\cite{sdp}. I will first discuss how to model using the SDP, then examine
properties of the model through a data analysis.

\section{Spatial Dirichlet Process Modeling}
\noindent
To denote realizations from point-referenced spatial data, we use $\{y(s): s\in
S\}, S \subset R^d$, where $d$ is the dimension of $S$. We observe data only at
a subset of all possible points in $S$, $\m s^{(n)}=(s_1,...,s_n)$ . Typically,
this type of data is modeled by a Guassian process (GP). However, the
assumption that the data arises from a GP is often a restrictiveon.  we may
want to allow deviation from a Gaussian random field. An SDP prior can be put
on the random field and have a Gaussian process as the baseline distribution.
Required for the model are replicates at each point. That is we need the full
dataset to consist of a collection of vectors $\m y_t = (y(s_1),...,y(s_n))$,
$t=1,,,.T$. Note that the points $s_i$ can be a pair of latitudes and
longitudes.\\

\noindent
We can construct the model as follows:
\[
  \begin{array}{rclcl}
    %\m y_t &|& \m\theta_t,\beta,\tau^2 &\overset{ind.}{\sim}&\text{Normal}_n(\m\theta_t+ \m x_t\beta,
    \m y_t &|& \m\theta_t,\beta,\tau^2 &\overset{ind.}{\sim}&\text{N}_n(\m\theta_t+ \m{1_n}\beta,
    ~\tau^2\m I_n), ~~_{t=1,...,T}\\
    \m\theta_t &|& G^{(n)} &\overset{i.i.d.}{\sim}& G^{(n)}, ~~_{t=1,...,T} \\
    G^{(n)} &|& \alpha, \sigma^2, \phi &\sim&
      \text{DP}(~\alpha,G_0^{(n)}~\text{N}_n(\m 0_n,\sigma^2H_n(\phi)) ~) \\
    \\
            && \beta, \tau^2 &\sim& \text{N}(m,s^2) \times \text{IGamma}(a_{\tau^2}=2,b_{\tau^2}) \\
            && \alpha &\sim& \text{Gamma}(a_\alpha,b_\alpha) \\
            && \sigma^2 &\sim& \text{IGamma}(b_{\sigma^2}=2,b_{\sigma^2}) \\
            && \phi &\sim& \text{Uniform}(0,b_\phi) \\
  \end{array}
\]
where $H_n(\phi)$ is a covariance function, for example, the exponential
covariance function with decay parameter $\phi$. (i.e.  $(H_n(\phi))_{ij} =
\exp\left\{-\phi~\norm{s_i-s_j}\right\}$.)\\

\noindent
Here, $\m\theta_t$ are the location-specific mean deviations from a grand mean
$\beta$ across the $n$ spatial locations. Notice that clustering can result
among the $\m\theta_t$'s. This may be useful when our $\theta_t$'s are indexed
by time and we want to learn how the observations are clustered in time.  Also
notice that if we replaced the prior for the $\m\theta$ the baseline
distribution used, we get a Gaussian process.  And as $\alpha \rightarrow
\infty$, $\m\theta_t$ become i.i.d. $G_0^{(n)}$ conditional on the
hyperparameters. We assume $\m y_t$ to be independent and multivariate
normal with no correlation. That is, the covariance is $\tau^2 \m I_n$.
The $\m y_t$'s are simply modeled as a mixture of multivariate normals.\\

\subsection{Prior Specification}
The grand mean $\beta$ modeled with a normal prior. In modeling maximum
temperatures, it is appropriate to use a reasonably informative prior. In the
following data analysis, a prior mean of 30 and a prior variance of 5 were used.

\subsection{Posteriors}
\[\def\arraystretch{1.4}
  \begin{array}{rclcl}
    \beta &|& \m y, \m\theta, \tau^2  &\sim& \text{N}(
      \frac{\tau^2m + s^2\sum_{t=1}^T\sum_{i=1}^n(y_{it}-\theta_{it})}{\tau^2+Tns^2},
      \frac{s^2\tau^2}{\tau^2+Tns^2}) \\
    \tau^2 &|& \m y, \m\theta, \beta  &\sim& \text{IG}(a_{\tau^2}+\frac{nT}{2},
    b_{\tau^2}+\frac{\sum_{t=1}^T (\m\mu_t-\beta \m1_n)'(\m\mu_t-\beta \m1_n)}{2})\\
    %p(\sigma^2 &|& \m\theta_t^*, T^*, \m y, \sigma^2) &\propto& [\sigma^2]\prod_{t=1}^{T^*}
    %  ~\text{N}_n(\m\theta_t^* | \m 0_n,\sigma^2H_n(\phi)) ~)\\
    \sigma^2 &|& \m\theta^*, T^*, \m y, \sigma^2 &\sim& \text{IG}(a_{\sigma^2}+\frac{nT^*}{2},
      %b+\frac{\sum_{t=1}^{T^*}\m\theta^*'Hn^{-1}(\phi)\m\theta^*}{2})\\
    b_{\sigma^2}+\frac{\sum_{t=1}^{T^*}\m\theta_t^{*'} H_n^{-1}(\phi)\m\theta_t^*}{2})\\
    %p(\phi &|& \m\theta_t^*, T^*, \m y, \sigma^2) &\propto& \prod_{t=1}^{T^*} 
    %  ~\text{N}_n(\m\theta_t^* |\m 0_n,\sigma^2H_n(\phi)) ~)\\
    p(\phi &|& \m\theta^*, T^*, \m y, \sigma^2) &\propto& 
    [\phi]|H_n(\phi)|^{-T^*/2} \exp\left(-\frac{\sum_{t=1}^{T^*}\m\theta_t^{*'} H_n^{-1}(\phi)\m\theta_t^*}{2\sigma^2}\right)\\
    \eta &|& \alpha, \m y &\sim& \text{Beta}(\alpha+1,T)\\
    p(\alpha &|& T^*, \m y) &=& (\epsilon)~\gamma(\alpha | a_\alpha+T^*, b_\alpha-\log(\eta)) +\\
             &&&&(1-\epsilon)~\gamma(\alpha | a_\alpha+T^*-1, b_\alpha-\log(\eta))\\
  \end{array}
\]
where $\m\mu_t=\m{y_t -\theta_t} $, $\eta$ is an auxiliary variable introduced
to make the prior for $\alpha$ conjugate, $\gamma$ is the gamma density
function with the mean and rate parameterization, and 
$\epsilon = \ds\frac{a_\alpha +T^* - 1}{ n(b_\alpha-\log(\eta)) + a_\alpha +T^* -1}$.\\

\noindent
As $\m\theta_t$ has prior conjugacy, we can use the algorithm provided by \cite{escobar}
\subsection{Posterior for $\m\theta$}
\def\mm{\m{y_t}-\m{1_n}\beta}
\[\def\arraystretch{1.4}
  \begin{array}{rllcl}
    \m\theta_t &|& y_t,\beta,\tau^2,\sigma^2,\phi &\sim& \text{N}_n(\tau^{-2}\m\Lambda(\mm), \m\Lambda)\\
               && q0 &=& |\Lambda|^{1/2} \\
               %&&&& \times \exp\left\{.5\tau^{-2}(\mm)'(\m{I_n}-\tau^{-2}\m\Lambda)(\mm)\right\}\\
               &&&& \times \exp\left\{\frac{(\mm)'(\m{I_n}-\tau^{-2}\m\Lambda)(\mm)}{2\tau^2}\right\}\\
               &&&& \times [(2\pi\tau^2\sigma^2)^{n/2} |\m H_n(\phi)|^{1/2}]^{-1}\\
  \end{array}
\]
where $\m\Lambda = [\tau^{-2}\m I_n + \sigma^{-2} \m H_n^{-1}(\phi)]^{-1}$.


\section{Data Analysis}
\beginmyfig 
  \includegraphics[scale=.5]{../graphs/july1989.pdf} 
  \caption{Average maximum daily temperatures at various locations in the state of California in 1989. 
  Warmer areas are dark red and cooler locations are dark blue.}
  \label{fig:dat}
\endmyfig

\beginmyfig
  \includegraphics[scale=.5]{../graphs/postpredmean.pdf}
  \caption{}
  \label{fig:postpred}
\endmyfig

\beginmyfig
  \includegraphics[scale=.5]{../graphs/postpredvar.pdf}
  \caption{}
  \label{fig:datmean}
\endmyfig

%%% Bibliography
\bibliographystyle{asabyu} % bibtex
\bibliography{sdp}         % bibtex
\end{document}
